% SHIFT - project abstract

\documentclass{sig-alternate}

\usepackage{url}
\usepackage{color}
\usepackage{enumerate}
\usepackage{balance}
\usepackage{verbatim}
\permission{}
\CopyrightYear{2012}
%\crdata{0-00000-00-0/00/00}
\begin{document}

\title{SHIFT - Secure Heterogeneous InFormation Transfer for Relational Databases}
\numberofauthors{1}
\author{
\alignauthor{
Christopher A. Wood, TODO, TODO \\
caw4567@rit.edu, TODO, TODO
}}
\date{26 November 2012}
\maketitle
\begin{abstract}

\begin{comment}
  The abstract should be one or two paragraphs that
  summarize your paper. Abstracts are read independently
  from the rest of the paper so you cannot cite your paper
  or other papers in it. Study other abstracts in the papers
  you are reading to understand what an abstract should
  really means. Write the abstract in third person.
  
  TODO: Submit a 1-page description of your project including your sample database 
application, and the security features to be explored. This must be the abstract 
part of a formal research paper using the ACM style LaTeX template (see
myCourses); you will fill in the rest of this report for future phases.

TODO:
1. define the need for the system
2. define how we attempt to extend past solutions with new solution
3. define rationale for JSON/PKI
\end{comment}

Data exchange between organizations is becoming a pervasive problem in the computer
security landscape, particularly in the context of health information systems. With the 
timely access to accurate data, health practitioners are able to make informed decisions
about patient treatments. Such data access is particularly important in the cardiovascular domain,
which, according to the World Health Organization (WHO), is the primary cause of 
individual fatalities in developed countries. Access to a patient's medical history 
is critically important in order to successfully diagnose a wide array of cardiovascular problems (CITE \#3/4 from paper).

Standard solutions for secure information exchange, such as the Electronics Data Interchange (EDI),
have been deployed in clinical settings for several years. However, with the emergence of modern
web applications and services, the adoption of solutions based on eXtensible Markup Language (XML)
technologies have risen in popularity. With eXtensible Stylesheet Language Transformations (XSLTs), 
XML documents containing vital data from a separate organization can be molded to match a different, yet 
compatible, schema. The interoperability properties of XML have thus given rise to data interchange frameworks
based on XML solutions in recent years \cite{Jumaa10-XmlExchange}.

JavaScript Object Notation, or JSON, is another increasingly popular data interchange format. While it is
similar in XML in many regards, its syntactic simplicity makes it a very appealing alternative to XML for
data exchange (TODO: cite JSON). Since JSON is not a document markup language, it does not have the
same extensibility of  XML. However, its flat structure enables much easier and more efficient parsing of 
data, and thus makes it an appealing candidate for addressing the problem of secure data exchange.

SHIFT, a Secure Heterogeneous Information Transfer mechanism for relational databases, is inspired 
by the mediator design pattern for centralizing and managing pairwise interactions between many 
subjects using a publish-subscribe enrollment approach. Subjects will register with the SHIFT service
by providing their own database schema and other pertinent identification information. This will enable
SHIFT to push and pull data from subject databases using their provided schema. Once registered,
subjects can push data to other known subjects by transferring data to the SHIFT service. Similarly,
subjects can query for data from other known subjects by requesting data from the SHIFT service. 

This framework builds upon the model presented in \cite{Jumaa10-XmlExchange} with two very
important enhancements. Firstly, JSON, rather than XML, 
is used as the data interchange format between relational databases. Mappings between database schemas, 
as well as the schemas themselves, can easily be represented using JSON. Secondly, the notion of data
confidentiality while transmissions are being made is not discussed in \cite{Jumaa10-XmlExchange}. Therefore, 
we plan on extending the framework to include PKI system for encryption. A CA is used to sign all certificates.

TODO: need two more paragraphs.
  
\end{abstract}

\begin{comment}
\begin{table}
\centering
\caption{Feelings about Issues}
\begin{tabular}{|l|r|l|} \hline
Flavor&Percentage&Comments\\ \hline
Issue 1 &  10\% & Loved it a lot\\ \hline
Issue 2 &  20\% & Disliked it immensely\\ \hline
Issue 3 &  30\% & Didn't care one bit\\ \hline
Issue 4 &  40\% & Duh?\\ \hline
\end{tabular}
\end{table}

\begin{figure}[htb]
\label{sample graphic}
\begin{center}
\includegraphics[width=1.5in]{fly.jpg}
\caption{A sample black \& white graphic (JPG).}
\end{center}
\end{figure}
\end{comment}

\bibliographystyle{abbrv}
\bibliography{../shift}
% You must have a proper ".bib" file
%  and remember to run:
% latex bibtex latex latex
% to resolve all references
\balance
\end{document}








