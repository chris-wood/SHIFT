% SHIFT - project abstract

\documentclass{sig-alternate}

\usepackage{url}
\usepackage{color}
\usepackage{enumerate}
\usepackage{balance}
\usepackage{verbatim}
\permission{}
\CopyrightYear{2012}
%\crdata{0-00000-00-0/00/00}
\begin{document}

\title{SHIFT - Secure Heterogeneous InFormation Transfer for Relational Databases}
\numberofauthors{1}
\author{
\alignauthor{
Christopher A. Wood, TODO, TODO \\
caw4567@rit.edu, TODO, TODO
}}
\date{26 November 2012}
\maketitle
\begin{abstract}
  The abstract should be one or two paragraphs that
  summarize your paper. Abstracts are read independently
  from the rest of the paper so you cannot cite your paper
  or other papers in it. Study other abstracts in the papers
  you are reading to understand what an abstract should
  really means. Write the abstract in third person.
  
  TODO: Submit a 1-page description of your project including your sample database 
application, and the security features to be explored. This must be the abstract 
part of a formal research paper using the ACM style LaTeX template (see
myCourses); you will fill in the rest of this report for future phases.

TODO:
0. find the bibtex citation for the base paper
1. define the need for the system
2. define how we attempt to extend past solutions with new solution
3. define rationale for json/PKI
  
\end{abstract}

\begin{comment}
\begin{table}
\centering
\caption{Feelings about Issues}
\begin{tabular}{|l|r|l|} \hline
Flavor&Percentage&Comments\\ \hline
Issue 1 &  10\% & Loved it a lot\\ \hline
Issue 2 &  20\% & Disliked it immensely\\ \hline
Issue 3 &  30\% & Didn't care one bit\\ \hline
Issue 4 &  40\% & Duh?\\ \hline
\end{tabular}
\end{table}

\begin{figure}[htb]
\label{sample graphic}
\begin{center}
\includegraphics[width=1.5in]{fly.jpg}
\caption{A sample black \& white graphic (JPG).}
\end{center}
\end{figure}
\end{comment}

\bibliographystyle{abbrv}
\bibliography{../shift}
% You must have a proper ".bib" file
%  and remember to run:
% latex bibtex latex latex
% to resolve all references
\balance
\end{document}








